% !TEX TS-program = pdflatex
% !TEX encoding = UTF-8 Unicode

% This is a simple template for a LaTeX document using the "article" class.
% See "book", "report", "letter" for other types of document.

\documentclass[11pt]{article} % use larger type; default would be 10pt

\usepackage[utf8]{inputenc} % set input encoding (not needed with XeLaTeX)

%%% Examples of Article customizations
% These packages are optional, depending whether you want the features they provide.
% See the LaTeX Companion or other references for full information.

%%% PAGE DIMENSIONS
\usepackage{geometry} % to change the page dimensions
\geometry{a4paper} % or letterpaper (US) or a5paper or....
% \geometry{margin=2in} % for example, change the margins to 2 inches all round
% \geometry{landscape} % set up the page for landscape
%   read geometry.pdf for detailed page layout information

\usepackage{graphicx} % support the \includegraphics command and options

% \usepackage[parfill]{parskip} % Activate to begin paragraphs with an empty line rather than an indent

%%% PACKAGES
\usepackage{booktabs} % for much better looking tables
\usepackage{array} % for better arrays (eg matrices) in maths
%\usepackage{paralist} % very flexible & customisable lists (eg. enumerate/itemize, etc.)
\usepackage{verbatim} % adds environment for commenting out blocks of text & for better verbatim
\usepackage{subfig} % make it possible to include more than one captioned figure/table in a single float
\usepackage{setspace} %paquete para interlineado
% These packages are all incorporated in the memoir class to one degree or another...

%%% HEADERS & FOOTERS
\usepackage{fancyhdr} % This should be set AFTER setting up the page geometry
\pagestyle{fancy} % options: empty , plain , fancy
\renewcommand{\headrulewidth}{0pt} % customise the layout...
\lhead{}\chead{}\rhead{}
\lfoot{}\cfoot{\thepage}\rfoot{}

%%% SECTION TITLE APPEARANCE
\usepackage{sectsty}
\allsectionsfont{\sffamily\mdseries\upshape} % (See the fntguide.pdf for font help)
% (This matches ConTeXt defaults)

%%% ToC (table of contents) APPEARANCE
\usepackage[nottoc,notlof,notlot]{tocbibind} % Put the bibliography in the ToC
\usepackage[titles,subfigure]{tocloft} % Alter the style of the Table of Contents
\renewcommand{\cftsecfont}{\rmfamily\mdseries\upshape}
\renewcommand{\cftsecpagefont}{\rmfamily\mdseries\upshape} % No bold!

%%% END Article customizations

\usepackage[spanish]{babel}
\usepackage{listings} 
%%% The "real" document content comes below...

\title{Investigación de Lenguajes - Pascal}
\author{Javier Tibau}
%\date{} % Activate to display a given date or no date (if empty),
         % otherwise the current date is printed 

\begin{document}
\maketitle
%\tableofcontents % No hace falta un TOC en un artículo corto

\section{Introducción}
HTML (Hyper Text Markup Language), es un lenguaje utilizado para crear paginas web, con el se escriben muchas de las paginas web, que existen actualmente en el internet.\\

Los programadores utilizan el lenguaje HTML para crear sus páginas web, los programas que utilizan los programadores generan páginas escritas en HTML y los navegadores que utilizamos los usuarios muestran las páginas web después de leer su contenido HTML.\\
El lenguaje HTML fue estandarizado por el organismo W3C (World Wide Web Consortium) el cual define un estandar para todas las empresas relacionadas con el uso del internet. De esta manera el lenguaje HTML es un lenguaje universalmente reconocido y el cual permite publicar informacion de manera global.\\

Desde su creación, el lenguaje HTML ha pasado de ser un lenguaje utilizado exclusivamente para crear documentos electrónicos a ser un lenguaje que se utiliza en muchas aplicaciones electrónicas como buscadores, tiendas online y banca electrónica.\\

El eje principal de HTML es la referenciacion , es decir para insertar contenido multimedia como fotos, musica, videos, archivos, etc , estos no van dentro del codigo de la pagina, sino que se hace referencia a los mismos que se encuentran en una ubicacion externa. El navegador que interpreta el codigo HTML une todos estos archivos mostrando asi, una pagina final, mientras que el proceso de creacion del codigo HTML contiene unicamente texto. Ya que este es un lenguaje estandar es necesario que todos los navegadores ejecuten de la misma manera el codigo.\\

A lo largo del desarrollo de HTML, este ha incorporado diferentes caracteristicas para adaptarlo a nuevas plataformas (smartphones, tablet, etc), Estos cambios son añadidos a los navegadores cada vez que estos lanzan una nueva version o actualizacion. Para la correcta interpretacion de una nueva version de HTML es necesario que los navegadores esten actualizados, por el contrario las versiones actuales de navegadores por lo general mantienen la interpretacion de versiones anteriores de HTML, para poder visualizar el contenido de aplicaciones desarrolladas por las mismas, aunque la apariencia no sea moderna.\\

\section{Características}
\section{Historia}
\section{Tutorial de Instalación}
Para escribir HTML lo único que se necesita es un editor de texto ASCII, como EDIT del MS-DOS o el Bloc de notas de Windows. \singlespace
\thinspace Existen una serie de programas que ayudan en la elaboración de documentos HTML, como HTMLED (shareware) o HTML Assistant, ambos para Windows, pero no son imprescindibles para escribir el código. Lo que si es necesario es un programa cliente WWW, tal como Mosaic, o Netscape, para probar el documento a medida que lo vamos desarrollando.

\section{Ejemplos del Codigo:}

\lstset{language=Pascal}          % Set your language (you can change the language for each code-block optionally)

\begin{lstlisting}[frame=single]  % Start your code-block
for i:=maxint to 0 do
begin
{ do nothing }
end;
Write('Case insensitive ');
Write('Pascal keywords.');
\end{lstlisting}



\end{document}
